\documentclass[12pt]{article}
 
\usepackage[margin=1in]{geometry}
\usepackage{amsmath,amsthm,amssymb}
\usepackage{fancyhdr}
\usepackage{hyperref}
\pagestyle{fancy}

\newcommand{\N}{\mathbb{N}}
\newcommand{\R}{\mathbb{R}}
\newcommand{\Z}{\mathbb{Z}}
\newcommand{\Q}{\mathbb{Q}}
 
\newenvironment{theorem}[2][Theorem]{\begin{trivlist}
\item[\hskip \labelsep {\bfseries #1}\hskip \labelsep {\bfseries #2.}]}{\end{trivlist}}
\newenvironment{lemma}[2][Lemma]{\begin{trivlist}
\item[\hskip \labelsep {\bfseries #1}\hskip \labelsep {\bfseries #2.}]}{\end{trivlist}}
\newenvironment{exercise}[2][Exercise]{\begin{trivlist}
\item[\hskip \labelsep {\bfseries #1}\hskip \labelsep {\bfseries #2.}]}{\end{trivlist}}
\newenvironment{problem}[2][Problem]{\begin{trivlist}
\item[\hskip \labelsep {\bfseries #1}\hskip \labelsep {\bfseries #2.}]}{\end{trivlist}}
\newenvironment{question}[2][Question]{\begin{trivlist}
\item[\hskip \labelsep {\bfseries #1}\hskip \labelsep {\bfseries #2.}]}{\end{trivlist}}
\newenvironment{corollary}[2][Corollary]{\begin{trivlist}
\item[\hskip \labelsep {\bfseries #1}\hskip \labelsep {\bfseries #2.}]}{\end{trivlist}}
 
\begin{document}
 
\title{Lab 3: File Servers, Traffic Generators, and Monitoring}
\author{Duc Viet Le\\ 
CS536}
 
\maketitle
 
\begin{problem}{1} \ \\
For this problem, I use an audio file (i.e.) with file size of 3183357 bytes ($\approx 3.1$ MB). We use both check sum and Linux \texttt{diff} to check for the authenticity of the downloaded files. With TCP, we did not have any corrupted files. 
\\
Below are the results of time and through put for different package sizes:
\begin{center}
	\begin{tabular}{|c|c|c|}
	\hline 
	Package size & Time & Reliable Through Put \\ \hline
	1 byte & 5265 ms & 4,836,939 bps \\ \hline
	10 bytes & 590 ms & 43,152,693 bps \\ \hline
	50 bytes & 276 ms & 92,095,696 bps \\ \hline
	500 bytes & 274 ms & 92,382,231 bps \\ \hline
	1000 bytes & 273 ms & 93,115,342 bps \\ \hline
	1500 bytes & 272 ms & 92,695,123 bps \\ \hline
	2500 bytes & 273 ms & 93,255,654 bps \\ \hline
	10000 bytes & 275 ms & 92,895,832 bps \\ \hline
	\end{tabular}
\end{center}
Below is plotted graph of the datas:
\begin{center}
	
\end{center}
Package size did not have large impact on throughput as the package size increases. The reliable throughput and transmission time did not change. Thus, to improve the server performance, I may choose package size to be power of two because it may be helpful for computers and kernels, not too big so that we can fit inside processor (i.e L2 cache) ... 
\end{problem}
\begin{problem}{2}
\end{problem}
\begin{problem}{3}
\end{problem}
\end{document}
