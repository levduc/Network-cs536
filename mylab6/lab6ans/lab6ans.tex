\documentclass[12pt]{article}
 
\usepackage[margin=1in]{geometry}
\usepackage{amsmath,amsthm,amssymb}
\usepackage{fancyhdr}
\usepackage{hyperref}
\pagestyle{fancy}
\usepackage{graphicx}
\newcommand{\N}{\mathbb{N}}
\newcommand{\R}{\mathbb{R}}
\newcommand{\Z}{\mathbb{Z}}
\newcommand{\Q}{\mathbb{Q}}
 
\newenvironment{theorem}[2][Theorem]{\begin{trivlist}
\item[\hskip \labelsep {\bfseries #1}\hskip \labelsep {\bfseries #2.}]}{\end{trivlist}}
\newenvironment{lemma}[2][Lemma]{\begin{trivlist}
\item[\hskip \labelsep {\bfseries #1}\hskip \labelsep {\bfseries #2.}]}{\end{trivlist}}
\newenvironment{exercise}[2][Exercise]{\begin{trivlist}
\item[\hskip \labelsep {\bfseries #1}\hskip \labelsep {\bfseries #2.}]}{\end{trivlist}}
\newenvironment{problem}[2][Problem]{\begin{trivlist}
\item[\hskip \labelsep {\bfseries #1}\hskip \labelsep {\bfseries #2.}]}{\end{trivlist}}
\newenvironment{question}[2][Question]{\begin{trivlist}
\item[\hskip \labelsep {\bfseries #1}\hskip \labelsep {\bfseries #2.}]}{\end{trivlist}}
\newenvironment{corollary}[2][Corollary]{\begin{trivlist}
\item[\hskip \labelsep {\bfseries #1}\hskip \labelsep {\bfseries #2.}]}{\end{trivlist}}
 
\begin{document}
 
\title{Lab 6}
\author{Duc Viet Le\\
 CS536}
 
\maketitle
 
\begin{problem}{1} \ \\
The result is similar to the result of lab4:
\begin{itemize}
    \item For \texttt{myping.c}: since it takes longer for udp packets to travel between overlay nodes, the ping result increases as I increase the number of overlay nodes. I used 3 overlay routers which are \texttt{sslab02, sstlab04, sslab06}, and I run \texttt{mypingd.c} at \texttt{sslab01} and \texttt{myping.c} at \texttt{sslab08}. The ping results are:
    \begin{verbatim}
       1.997 ms
       2.225 ms
       1.792 ms
       2.132 ms
       2.023 ms
       1.989 ms
    \end{verbatim}
    Ping result without using overlay router is: \texttt{0.447 ms} on average.
    \item For \texttt{traffic\_send.c}: Since we test the app on multiple sslabs with pretty much bandwidth, there is no significant differences when using overlay routers. However, if there is a bottle neck node in one of the overlay node, I would expect the bandwidth to be decrease. Below is result using same overlay nodes (i.e \texttt{sslab02, sslab04, sslab06})
    \texttt{traffic\_snd} :
    \begin{verbatim}
        │Portnumber: 21806
        │payloadSize: 1000
        │Package Count: 1000
        │Package Spacing: 1000
        │Completion Time: 1.132008 s
        │Package Per Second (PPS): 883.386047 packages/s
        │Bit sent: 8440000
        │Bits Per Second (BPS): 7455778.000000 bps 
    \end{verbatim}
    \texttt{traffic\_rcv}
    \begin{verbatim}
        Port Number: 30000
        payloadSize: 1000
        Start listening ... 
        First Package arrived.
        End of transmission
        Package Count: 1000
        Completion Time: 1.133577 s
        Bits received: 8440000
        Package Per Second (PPS): 883.722412 packages/s
    \end{verbatim}
    \item For testing multiple clients, I do not see any significant changes. 
\end{itemize}

\end{problem}

\begin{problem}{2}
    
\end{problem}
\end{document}
