\documentclass[11pt]{article}
\usepackage[margin=1in]{geometry}
\usepackage{amsmath,amsthm,amssymb}
\usepackage{fancyhdr}
\usepackage{hyperref}
\usepackage{graphicx}
\newcommand{\N}{\mathbb{N}}
\newcommand{\R}{\mathbb{R}}
\newcommand{\Z}{\mathbb{Z}}
\newcommand{\Q}{\mathbb{Q}}
\pagestyle{fancy}
 
\newenvironment{theorem}[2][Theorem]{\begin{trivlist}
\item[\hskip \labelsep {\bfseries #1}\hskip \labelsep {\bfseries #2.}]}{\end{trivlist}}
\newenvironment{lemma}[2][Lemma]{\begin{trivlist}
\item[\hskip \labelsep {\bfseries #1}\hskip \labelsep {\bfseries #2.}]}{\end{trivlist}}
\newenvironment{exercise}[2][Exercise]{\begin{trivlist}
\item[\hskip \labelsep {\bfseries #1}\hskip \labelsep {\bfseries #2.}]}{\end{trivlist}}
\newenvironment{problem}[2][Problem]{\begin{trivlist}
\item[\hskip \labelsep {\bfseries #1}\hskip \labelsep {\bfseries #2.}]}{\end{trivlist}}
\newenvironment{question}[2][Question]{\begin{trivlist}
\item[\hskip \labelsep {\bfseries #1}\hskip \labelsep {\bfseries #2.}]}{\end{trivlist}}
\newenvironment{corollary}[2][Corollary]{\begin{trivlist}
\item[\hskip \labelsep {\bfseries #1}\hskip \labelsep {\bfseries #2.}]}{\end{trivlist}}
 
\begin{document}
\title{Assignment 3}
\author{Duc Viet Le\\ CS536}
 
\maketitle

\begin{problem}{1}
\end{problem}
\begin{enumerate}
	\item[a.] In the second segment sent from Host A to B:
	\begin{itemize}
		\item Sequence Number: $127+80=207$ 
		\item Source-Port: 302
		\item Destination-Port: 80
	\end{itemize}
	\item[b.] If the first segment arrives before the second segment, in the acknowledgment of the first arriving segment:
	\begin{itemize}
		\item Sequence Number: 207
		\item Source-Port: 80
		\item Destination-Port: 302
	\end{itemize}
	\item[c.] If the second segment arrive before the first segment, in the acknowledgment of the first arriving segment:
	\begin{itemize}
		\item Sequence Number: 127 
	\end{itemize}
	to tell $A$ that it has received everything up to 126.
	\item[d.] Draw:
	\pagebreak
\end{enumerate}

\begin{problem}{2}
\end{problem}
\begin{enumerate}
	\item[a.] The intervals of the time when TCP slow start is operating is $[1,6]\cup[23,26]$
	\item[b.] The intervals of the time when TCP congestion avoidance is operating is $[6,16]\cup [17,22]$
	\item[c.] By a triple duplicate ACK. Otherwise, the $cwnd$ would drop to 1
	\item[d.] By a timeout as the $cwnd$ dropped to 1
	\item[e.] $ssthresh= 32$ because TCP congestion avoidance starts at that point
	\item[f.] $ssthresh\approx 42/2 = 21$ since the $cwnd$ is around $24$ in the next round. 
	\item[g.] $ssthresh \approx 28/2 = 14$ since there is a time out when $cwnd$ is around $28$.
	\item[h.] During $7^{th}$ round. As we can see, during slow start phase (i.e. round 1-6), there are 63 segments sent. the $64-96^{th}$ segments are sent in the next round (i.e. $7^{th}$ round).
	\item[i.] $ssthresh=4$ which is half of the current $cwnd$. And the $cwnd$ for the next round is $ssthresh+3=7$ as TCP enters the fast retransmit phase.
	\item[j.] Suppose TCP Tahoe:
	\begin{itemize}
		\item $ssthresh = 42/2 = 21$
		\item $cwnd$ = 4. Since the TCP slow-start begins at round $17^{th}$, after two round, the $cwnd$ has to be $4$.
	\end{itemize}
	\item[k.] Suppose TCP Tahoe
	\begin{itemize}
		\item Round $17^{th}$ : 1 packets
		\item Round $18^{th}$ : 2 packets
		\item Round $19^{th}$ : 4 packets
		\item Round $20^{th}$ : 8 packets
		\item Round $21^{st}$ : 16 packets
		\item Round $22^{nd}$ : 21 packets (this is the current $ssthresh$)
	\end{itemize}
	Therefore, total packets sent are 52 packets.
\end{enumerate}
\end{document}
