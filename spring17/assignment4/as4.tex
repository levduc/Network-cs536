\documentclass[11pt]{article}
\usepackage[margin=1in]{geometry}
\usepackage{amsmath,amsthm,amssymb}
\usepackage{fancyhdr}
\usepackage{hyperref}
\usepackage{graphicx}
\newcommand{\N}{\mathbb{N}}
\newcommand{\R}{\mathbb{R}}
\newcommand{\Z}{\mathbb{Z}}
\newcommand{\Q}{\mathbb{Q}}
\pagestyle{fancy}
 
\newenvironment{theorem}[2][Theorem]{\begin{trivlist}
\item[\hskip \labelsep {\bfseries #1}\hskip \labelsep {\bfseries #2.}]}{\end{trivlist}}
\newenvironment{lemma}[2][Lemma]{\begin{trivlist}
\item[\hskip \labelsep {\bfseries #1}\hskip \labelsep {\bfseries #2.}]}{\end{trivlist}}
\newenvironment{exercise}[2][Exercise]{\begin{trivlist}
\item[\hskip \labelsep {\bfseries #1}\hskip \labelsep {\bfseries #2.}]}{\end{trivlist}}
\newenvironment{problem}[2][Problem]{\begin{trivlist}
\item[\hskip \labelsep {\bfseries #1}\hskip \labelsep {\bfseries #2.}]}{\end{trivlist}}
\newenvironment{question}[2][Question]{\begin{trivlist}
\item[\hskip \labelsep {\bfseries #1}\hskip \labelsep {\bfseries #2.}]}{\end{trivlist}}
\newenvironment{corollary}[2][Corollary]{\begin{trivlist}
\item[\hskip \labelsep {\bfseries #1}\hskip \labelsep {\bfseries #2.}]}{\end{trivlist}}
 
\begin{document}
\title{Assignment 3}
\author{Duc Viet Le\\ CS536}
 
\maketitle

\begin{problem}{1}
\end{problem}
\begin{enumerate}
	\item All addresses must be located from 214.97.254.0/24
	\begin{itemize}
		\item Subnet A: \textbf{214.97.255.0/24}
		\\Give subnet A the last 8 bit (256 addresses)
		\\Use first 24 bit to identify A: 11010110 01100001 11111111
		\item Subnet B: \textbf{214.97.254.128/25}
		\\
		Give B the  last 7 bit (128 addresses)\\
		Use first 25 bit to identify B: 11010110 01100001 11111110 1
		\item Subnet C: \textbf{214.97.255.128/25}
		\\ Give C the  last 7 bit (128 addresses)\\
		Use first 25 bit to identify C: 11010110 01100001 11111111 1

		\item Subnet D: \textbf{214.97.254.0/31}
		\\ Give D the last 1 bit (2 addresses)
		\\ Use first 31 bit to identify D:
		11010110 01100001 11111110 0000000
		\item Subnet E: \textbf{214.97.254.2/31}
		\\ Give E the last 1 bit (2 addresses)
		\\ Use first 31 bit to identify E:
		11010110 01100001 11111110 0000001
		\item Subnet F: \textbf{214.97.254.6/31}
		\\ Give F the last 1 bit (2 addresses)
		\\ Use first 31 bit to identify F:
		11010110 01100001 11111110 0000011
	\end{itemize}
	\item Forwarding tables:
	\\ \textbf{Router 1:} \\ \\
	\begin{tabular}{|l|c|}
	\hline prefix match &	interface \\
	\hline 11010110 01100001 11111111 &	Subnet A \\
	\hline 11010110 01100001 11111110 0000011&	Subnet F \\
	\hline 11010110 01100001 11111110 0000000&	Subnet D \\
	\hline
	\end{tabular}
	\\
	\\ \textbf{Router 2:} \\\\
	\begin{tabular}{|l|c|}
	\hline prefix match &	interface \\
	\hline 11010110 01100001 11111110 0000011&	Subnet F \\
	\hline 11010110 01100001 11111110 0000001&	Subnet E \\
	\hline 11010110 01100001 11111111 1&	Subnet C \\
	\hline
	\end{tabular}
	\\\\ 
	\textbf{Router 3:} \\\\
	\begin{tabular}{|l|c|}
	\hline prefix match &	interface \\
	\hline 11010110 01100001 11111110 0000000&	Subnet D \\
	\hline 11010110 01100001 11111110 0000001&	Subnet E \\
	\hline 11010110 01100001 11111110 1&	Subnet B \\
	\hline
	\end{tabular}
	\\
	
\end{enumerate}
\begin{problem}{2}
\end{problem}
\begin{itemize}
	\item Only traffic arriving from hosts h1 and h6 should be delivered to hosts h3 or h4 (i.e., that arriving traffic from hosts h2 and h5 is blocked).
	\\
	\resizebox{\textwidth}{!}{
	\begin{tabular}[\textwidth]{|l|l|l|l|l|l|l|l|l|l|l|}
	\hline Ingess port & Mac src & Mac dst & Eth type & VLAND ID & IP Src & IP Dst & IP prot & TCPsport & TCP dport & Action \\
	\hline * & * & * & * & * & 10.1.0.1 & 10.2.0.*& * & * & * & allow \\
	\hline * & * & * & * & * & 10.3.0.6 & 10.2.0.*& * & * & * & allow \\
	\hline * & * & * & * & * & 10.1.0.2 & *& * & * & * & drop \\
	\hline * & * & * & * & * & 10.3.0.5 & *& * & * & * & drop \\
	\hline
	\end{tabular}}
	\begin{itemize}
		\item explicitly allow all traffic from h1 and h6 to h3 or h4
		\item drop traffic that comes from h2,h5.
	\end{itemize}
	\item Only TCP traffic is allowed to be delivered to host h3 or host h4 (i.e., that UDP traffic is blocked). \\
	\resizebox{\textwidth}{!}{
	\begin{tabular}[\textwidth]{|l|l|l|l|l|l|l|l|l|l|l|}
	\hline Ingess port & Mac src & Mac dst & Eth type & VLAND ID & IP Src & IP Dst & IP prot & TCPsport & TCP dport & Action \\
	\hline * & * & * & * & * & * & 10.2.0.*& TCP & * & * & allow \\
	\hline * & * & * & * & * & * & 10.2.0.*& UDP & * & * & drop \\
	\hline
	\end{tabular}}
	\begin{itemize}
		\item allow TCP trafffic that has destination to h3 or h4
		\item drop UDP traffic that go to h3 or h4
	\end{itemize}
	\item Only traffic destined to h3 is to be delivered (i.e., all traffic to h4 is blocked).\\
	\resizebox{\textwidth}{!}{
	\begin{tabular}[\textwidth]{|l|l|l|l|l|l|l|l|l|l|l|}
	\hline Ingess port & Mac src & Mac dst & Eth type & VLAND ID & IP Src & IP Dst & IP prot & TCPsport & TCP dport & Action \\
	\hline * & * & * & * & * & * & 10.2.0.3 & * & * & * & allow \\
	\hline * & * & * & * & * & * & 10.2.0.4 & * & * & * & drop \\
	\hline
	\end{tabular}}
	\begin{itemize}
		\item explicitly allow traffic that comes to h3.
		\item drop traffic to h4.
	\end{itemize}
	\item Only UDP traffic from h1 and destined to h3 is to be delivered. All other traffic is blocked.\\
	\resizebox{\textwidth}{!}{
	\begin{tabular}[\textwidth]{|l|l|l|l|l|l|l|l|l|l|l|}
	\hline Ingess port & Mac src & Mac dst & Eth type & VLAND ID & IP Src & IP Dst & IP prot & TCPsport & TCP dport & Action \\
	\hline * & * & * & * & * & 10.1.0.1 & 10.2.0.3& UDP & * & * & allow \\
	\hline * & * & * & * & * & * & 10.2.0.4& * & * & * & drop \\
	\hline * & * & * & * & * & * & *& TCP & * & * & drop \\
	\hline * & * & * & * & * & 10.1.0.2 &*& * & * & * & drop \\
	\hline * & * & * & * & * & 10.2.0.* & * & * & * & * & drop \\
	\hline * & * & * & * & * & 10.3.0.* & * & * & * & * & drop \\
	\hline
	\end{tabular}}
	\begin{itemize}
		\item explicily allow UDP traffic from h1 to h3.   
		\item drop everything else
	\end{itemize}
\end{itemize}
\end{document}
