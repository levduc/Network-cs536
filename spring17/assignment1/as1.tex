\documentclass[12pt]{article}
 
\usepackage[margin=1in]{geometry}
\usepackage{amsmath,amsthm,amssymb}
\usepackage{fancyhdr}
\usepackage{hyperref}
\pagestyle{fancy}
\usepackage{graphicx}
\newcommand{\N}{\mathbb{N}}
\newcommand{\R}{\mathbb{R}}
\newcommand{\Z}{\mathbb{Z}}
\newcommand{\Q}{\mathbb{Q}}
 
\newenvironment{theorem}[2][Theorem]{\begin{trivlist}
\item[\hskip \labelsep {\bfseries #1}\hskip \labelsep {\bfseries #2.}]}{\end{trivlist}}
\newenvironment{lemma}[2][Lemma]{\begin{trivlist}
\item[\hskip \labelsep {\bfseries #1}\hskip \labelsep {\bfseries #2.}]}{\end{trivlist}}
\newenvironment{exercise}[2][Exercise]{\begin{trivlist}
\item[\hskip \labelsep {\bfseries #1}\hskip \labelsep {\bfseries #2.}]}{\end{trivlist}}
\newenvironment{problem}[2][Problem]{\begin{trivlist}
\item[\hskip \labelsep {\bfseries #1}\hskip \labelsep {\bfseries #2.}]}{\end{trivlist}}
\newenvironment{question}[2][Question]{\begin{trivlist}
\item[\hskip \labelsep {\bfseries #1}\hskip \labelsep {\bfseries #2.}]}{\end{trivlist}}
\newenvironment{corollary}[2][Corollary]{\begin{trivlist}
\item[\hskip \labelsep {\bfseries #1}\hskip \labelsep {\bfseries #2.}]}{\end{trivlist}}
 
\begin{document}
 
\title{Assignment 1}
\author{Duc Viet Le\\
 CS536}
 
\maketitle
 
\begin{problem}{1} \ \\
Suppose you would like to urgently deliver 40 terabytes data from Boston to Los Angeles. You
have available a 100 Mbps dedicated link for data transfer. Would you prefer to transmit the
data via this link or instead use FedEx over night delivery? Explain.
\end{problem}
\textbf{Ans. } Convert: $100$ Mbs = $10^8$ bps and $40$ terabytes = $4 \times 8 \times 10^{13}$ bits.
\\ By considering only the transmission time, we have: 
$$T = \frac{4 \times 8 \times 10^{13} } {10^8} = 32 \times 10^5\ (s) \approx 37.037\ (days)$$
Therefore, using FedEx over night delivery is a better solution. 	
\begin{problem}{2}
Suppose two hosts, A and B, are separated by $20,000$ kilometers and are connected by a direct
link of $R = 2$ Mbps. Suppose the propagation speed over the link is $2.5\times 10^8$ meters/sec.
\\
Convert: $20,000$ km $= 2\times 10^7$ m, $R = 2\ Mbps = 2 \times 10^6\ bps$
\begin{enumerate}
	\item[a.] Calculate the propagation delay, $d_{prop}$. 
	\\
	\textbf{Ans.} Using the equation: 
	$$d_{prop} = \frac{d}{s} = \frac{2\times 10^7\ m}{2.5 \times 10^8\ m/s} = 0.08\ s$$
	\item[b.] Bandwidth delay product,  $R \times d_{prop}$
	\\
	\textbf{Ans.} 
	$$R\times d_{prop} = 2\times 10^6 \times 0.08 = 1.6 \times 10^5\ bits$$
	\item[c.] Consider sending a file of $800,000$ bits from Host A to Host B. Suppose the file is sent	
	continuously as one large message. What is the maximum number of bits that will be
	in the link at any given time?
	\\
	\textbf{Ans.} Since $800,000 >> 1.6 \times 10^5$ and we don't mention about processing delay e.t.c, the maximum number of bits that will be in the link at any given time is $1.6 \times 10^5$ bits
	\item[d.]
	Provide an interpretation of the bandwidth-delay product.
	\\
	\textbf{Ans.} It's the maximum number of bits that can be in the link.
	\item[e.] What is the width (in meters) of a bit in the link? Is it longer than a football field?
	\textbf{Ans.} Since the maximum number of bits in the link is $1.6 \times 10^5$ bits, the width of a bit in the link is:
	$$\frac{2\times 10^7\ m}{1.6\times 10^5} = 125\ m$$
	Yes, it is bigger than a football field (i.e 109.1 m).
	\item[f.] Derive a general expression for the width of a bit in terms of the propagation speed $s$,
	the transmission rate $R$, and the length of the link $m$.
	\\
	\textbf{Ans.} Based on what we did, the equation for the the width of a bit is:
	$$\frac{distance}{bandwidth-delay\ product} = \frac{m}{R\times d_{prop}} = \frac{m}{R \times \frac{m}{s}} =  \frac{s}{R}$$
\end{enumerate}
\end{problem}
\end{document}
